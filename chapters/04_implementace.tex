\chapter{Implementace aplikace}
Tato kapitola popisuje implementaci navržené architektury a klíčových částí aplikace.
Vzhledem k rozsahu výsledného řešení se zaměřuje především na zásadní a pro tento systém specifické prvky, které mají největší význam z hlediska návrhu a struktury aplikace.

Pozornost je věnována zejména implementaci reprezentace výsledků operací, práci s geografickými dotazy nad databází a stránkování jejich výsledků, správě stavu uživatelského rozhraní a bezpečnostním mechanismům.
Vybrané části ilustrují, jak se návrhová rozhodnutí popsaná v předchozí kapitole promítla do konkrétní realizace aplikace.


\section{Implementace doménové vrstvy}

\subsection{Reprezentace výsledků operací}
% sealed class, functional přístup

\subsection{Implementace use-case tříd}
% řetězení volání repozitářů, řízení běhu use casu


\section{Implementace datové vrstvy}

\subsection{Mapování datových modelů a zpracování chyb}
% zachytávání výjimek při volání API a SDK a jejich převod na DomainResult a DomainError

\subsection{Geografické dotazy a stránkování}
% geohash

\subsection{Kategorie obchodů v NoSQL databázi}
% nemožnost joinů - samostatná entita a její aktualizování


\section{Implementace uživatelského rozhraní}

\subsection{Navigační graf aplikace}

\subsection{Správa stavu a vedlejší efekty}


\section{Bezpečnost a řízení přístupu}

\subsection{Firebase Security Rules}

\subsection{Validace dat a ochrana integrity}


\section{Shrnutí implementační části}

% 10-12 stran (TODO: smazat)