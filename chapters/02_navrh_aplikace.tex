\chapter{Návrh aplikace}

Tato kapitola se zabývá návrhem mobilní aplikace na základě výsledků provedené analýzy.
Cílem návrhu je definovat funkční rozsah aplikace, její strukturu, uživatelské rozhraní a technologické řešení tak, aby odpovídaly potřebám cílových uživatelů a zvolenému zaměření aplikace.
K tomuto byly využity standardní techniky softwarového inženýrství, zejména
specifikace požadavků a návrhové diagramy.

Nejprve jsou specifikovány funkční a nefunkční požadavky, které vymezují chování a vlastnosti aplikace.
Na jejich základě je dále vytvořen use case diagram znázorňující základní interakce mezi uživateli a aplikací.
Struktura aplikace je následně popsána pomocí class diagramu, který zachycuje hlavní entity a jejich vzájemné vztahy.
Součástí kapitoly je také návrh uživatelského rozhraní, jenž se zaměřuje na rozvržení obrazovek a způsob interakce s~uživatelem.
Závěrečná část kapitoly se věnuje výběru technologií, které budou vyhovovat požadavkům aplikace a umožní její efektivní implementaci.


\section{Specifikace požadavků}

Tato kapitola se zabývá specifikací požadavků na navrhovanou aplikaci.
Požadavky vymezují očekávané chování systému a jeho vlastnosti z pohledu uživatelů i provozu aplikace.

\subsection{Funkční požadavky}

Funkční požadavky popisují chování aplikace a funkce, které aplikace poskytuje svým uživatelům.
Aplikace rozlišuje dva základní typy uživatelů – běžné návštěvníky aplikace a registrované uživatele.

\begin{itemize}
    \item Uživatel může zobrazit obchody v~mapovém nebo seznamovém zobrazení.
    \item Uživatel může filtrovat zobrazené obchody podle kategorie, vzdálenosti a průměrného hodnocení.
    \item Uživatel může zobrazit detail obchodu, včetně popisu, fotografií a kontaktních údajů.
    \item Uživatel může vytvořit uživatelský účet.
    \item Registrovaný uživatel může vytvářet nové obchody.
    \item Registrovaný uživatel může upravovat a spravovat obchody, které vytvořil.
    \item Registrovaný uživatel může upravovat své kontaktní údaje zobrazované u~jeho obchodů.
    \item Registrovaný uživatel může ke svým obchodům přidávat název, popis, kategorie nabízených produktů, fotografie, nabídku produktů a otevírací dobu.
    \item Registrovaný uživatel může vytvářet recenze obchodů ostatních uživatelů.
\end{itemize}

Aplikace neslouží jako e-shop a neumožňuje přímý nákup nebo objednávky.
Jejím cílem je poskytnout prostředek pro prezentaci nabídky domácích producentů na mapě a umožnit zájemcům snadno vyhledat obchody v~jejich okolí.

\subsection{Nefunkční požadavky}


\section{Use case diagram}


\section{Class diagram}


\section{Návrh uživatelského rozhraní}


\section{Výběr technologií}
% Compose (vs Java/xml), Material 3, Firebase (vs Server Backend, vs Spring Server a proč jsem si to nevybral), Dokumentová databáze (vs SQL), BaaS, Serverless backend??
