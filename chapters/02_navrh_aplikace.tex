\chapter{Návrh aplikace}

Tato kapitola se zabývá návrhem mobilní aplikace na základě výsledků provedené analýzy.
Cílem návrhu je definovat funkční rozsah aplikace, její strukturu, uživatelské rozhraní a technologické řešení tak, aby odpovídaly potřebám cílových uživatelů a zvolenému zaměření aplikace.
K tomuto byly využity standardní techniky softwarového inženýrství, zejména
specifikace požadavků a návrhové diagramy.

Nejprve jsou specifikovány funkční a nefunkční požadavky, které vymezují chování a vlastnosti aplikace.
Na jejich základě je dále vytvořen use case diagram znázorňující základní interakce mezi uživateli a aplikací.
Struktura aplikace je následně popsána pomocí class diagramu, který zachycuje hlavní entity a jejich vzájemné vztahy.
Součástí kapitoly je také návrh uživatelského rozhraní, jenž se zaměřuje na rozvržení obrazovek a způsob interakce s~uživatelem.
Závěrečná část kapitoly se věnuje výběru technologií, které budou vyhovovat požadavkům aplikace a umožní její efektivní implementaci.


\section{Specifikace požadavků}

Tato kapitola se zabývá specifikací požadavků na navrhovanou aplikaci.
Požadavky vymezují očekávané chování systému a jeho vlastnosti z pohledu uživatelů i provozu aplikace.

\subsection{Funkční požadavky}

Funkční požadavky popisují chování aplikace a funkce, které aplikace poskytuje svým uživatelům.
Aplikace rozlišuje dva základní typy uživatelů – běžné návštěvníky aplikace a registrované uživatele.

\begin{itemize}
    \item Uživatel může zobrazit obchody v~mapovém nebo seznamovém zobrazení.
    \item Uživatel může filtrovat zobrazené obchody podle kategorie, vzdálenosti a průměrného hodnocení.
    \item Uživatel může zobrazit detail obchodu, včetně popisu, fotografií a kontaktních údajů.
    \item Uživatel může vytvořit uživatelský účet.
    \item Registrovaný uživatel může vytvářet nové obchody.
    \item Registrovaný uživatel může upravovat a spravovat obchody, které vytvořil.
    \item Registrovaný uživatel může upravovat své kontaktní údaje zobrazované u~jeho obchodů.
    \item Registrovaný uživatel může ke svým obchodům přidávat název, popis, kategorie nabízených produktů, fotografie, nabídku produktů a otevírací dobu.
    \item Registrovaný uživatel může vytvářet recenze obchodů ostatních uživatelů.
\end{itemize}

Aplikace neslouží jako e-shop a neumožňuje přímý nákup nebo objednávky.
Jejím cílem je poskytnout prostředek pro prezentaci nabídky domácích producentů na mapě a umožnit zájemcům snadno vyhledat obchody v~jejich okolí.

\subsection{Nefunkční požadavky}

Nefunkční požadavky popisují vlastnosti aplikace, které se netýkají přímo funkcionality, ale ovlivňují kvalitu, použitelnost a provoz aplikace.

\begin{itemize}
    \item Aplikace je určena pro platformu Android.
    \item Uživatelské rozhraní aplikace musí být přehledné a snadno použitelné i~pro méně technicky zdatné uživatele.
    \item Aplikace musí umožňovat plynulé zobrazení mapy a obchodů bez výrazných prodlev.
    \item Data uložená v~aplikaci musí být přístupná pouze oprávněným uživatelům.
    \item Aplikace musí být navržena tak, aby bylo možné ji dále rozšiřovat o~další funkce.
\end{itemize}


\section{Use case diagram}

Use case diagram slouží k~přehlednému znázornění základních funkcí aplikace a interakcí mezi uživateli a systémem.
Diagram vychází ze specifikace funkčních požadavků a zachycuje chování aplikace z~pohledu jednotlivých typů uživatelů.

V~rámci aplikace jsou rozlišeny tři základní role: obecného uživatele (User), zákazníka (Customer) a prodejce (Seller).
Obecný uživatel představuje nadřazenou roli, ze které vycházejí specializované role zákazníka a prodejce.
Tyto specializované role jsou rozděleny podle toho, jakým způsobem chce uživatel aplikaci využívat.

Obecný uživatel zahrnuje možnosti registrace, přihlášení a odhlášení z~aplikace, které jsou přístupné všem uživatelům bez ohledu na jejich způsob využití aplikace.
Umožňuje také úpravu osobních údajů.
Tato možnost je přístupná jakémukoli přihlášenému uživateli.

Zákazník může procházet obchody v~mapovém nebo seznamovém zobrazení, zobrazovat detail obchodu a filtrovat obchody podle kategorií, vzdálenosti nebo hodnocení.
Tyto filtry jsou v~diagramu modelovány jako rozšiřující případy užití nad základními způsoby procházení obchodů.
Pokud je běžný uživatel přihlášen, může vytvářet recenze obchodů ostatních uživatelů.

Druhým aktérem je prodejce, který může vytvářet, upravovat a odstraňovat své obchody.
V rámci správy obchodu může prodejce upravovat polohu obchodu na mapě, otevírací dobu, nabídku produktů a kategorie nabízených produktů.
Všechny interakce prodejce vyžadují přihlášení.

\begin{figure}[p]
    \centering
    \includegraphics[width=\textwidth]{res/use_case_diagram}
    \caption{Use case diagram aplikace}
    \label{fig:use-case-diagram}
\end{figure}
\FloatBarrier


\section{Class diagram}


\section{Návrh uživatelského rozhraní}


\section{Výběr technologií}
% Compose (vs Java/xml), Material 3, Firebase (vs Server Backend, vs Spring Server a proč jsem si to nevybral), Dokumentová databáze (vs SQL), BaaS, Serverless backend??
