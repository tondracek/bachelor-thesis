\chapter*{Úvod}
\addcontentsline{toc}{chapter}{Úvod}

V~posledních letech roste zájem zákazníků o~nákup potravin a výrobků od menších producentů, lokálních farem nebo domácích pěstitelů.\footfullcite{lokalni_potraviny_zajem, szif_2025}
Přestože se tento trend dále rozvíjí, dostupné nástroje pro prezentaci nabídky drobných a domácích producentů nejsou vždy zaměřeny na jednoduchou a samostatnou propagaci jejich produktů pro zákazníky v~okolí.

Lokální prodej je často založen na tradičních postupech, jako je například vyvěšení cedulí u~silnic nebo pouze osobní doporučení.
Takovéto metody mají často malý dosah a producentům i zákazníkům by mohl usnadnit vzájemné propojení nástroj, který by umožnil přehlednější a dostupnější prezentaci nabídky.

Cílem této práce je navrhnout a implementovat aplikaci pro platformu Android, která se snaží řešit uvedenou problematiku.
Aplikace umožní uživatelům vytvářet a spravovat vlastní obchody pro zobrazení jejich nabídky, zobrazovat obchody na mapě a filtrovat pomocí zvolených kritérií.

V~první kapitole práce je představena motivace pro tuto aplikaci a přehled již existujících podobných projektů.
Druhá kapitola popisuje návrh aplikace včetně specifikace požadavků, návrh diagramů a výběr technologií.
Třetí kapitola se zaměřuje na architektonická rozhodnutí, která usnadňují vývoj a údržbu aplikace.
Následuje implementační a testovací část.
Poslední kapitola popisuje proces vydání a nasazení aplikace pro platformu Android.
Práce je zakončena shrnutím dosažených výsledků a zhodnocením zvolených postupů.