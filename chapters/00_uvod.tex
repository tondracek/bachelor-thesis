\chapter*{Úvod}
% \markright{\textsc{Úvod}}
\addcontentsline{toc}{chapter}{Úvod}

V posledních letech roste zájem zákazníků o nákup potravin a výrobků od menších producentů, lokálních farem nebo domácích pěstitelů. Přestože se tento zájem rozvíjí, v mnoha případech chybí jednoduchý digitální nástroj, který by umožňoval těmto producentům snadno prezentovat svou nabídku a efektivně oslovovat zákazníky ve svém okolí.

Lokální prodej je často založen na tradičních a spíše neefektivních postupech, jako je například vyvěšení cedulí u silnic nebo pouze osobní doporučení. Takovéto metody mají často malý dosah a producentům i zákazníkům by velmi pomohl nástroj, který by zajistil lepší prezentaci jejich nabídky.

Cílem této práce je navrhnout a implementovat aplikaci pro platformu Android, která by tuto situaci pomohla vyřešit. Aplikace umožní uživatelům vytvářet a spravovat vlastní obchody pro zobrazení jejich nabídky, zobrazovat obchody na mapě a filtrovat pomocí zvolených kritérií.

V první kapitole práce je představena motivace pro tuto aplikaci a přehled již existujících podobných projektů. Druhá kapitola popisuje návrh aplikace včetně specifikace požadavků, návrh diagramů a výběr technologií. Třetí kapitola se zaměřuje na architektonická rozhodnutí, která usnadňují vývoj a údržbu aplikace. Následuje implementační a testovací část. Poslední kapitola popisuje proces vydání a nasazení aplikace pro platformu Android. Práce je zakončena shrnutím dosažených výsledků a zhodnocením zvolených postupů.