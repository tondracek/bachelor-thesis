\chapter{Analýza}
V~oblasti lokálního prodeje potravin a dalších produktů se malé farmy a domácí producenti často potýkají s~omezenými možnostmi, jak efektivně oslovit zákazníky ve svém okolí, bez nutnosti prodávat své produkty prostřednictvím velkých řetězců nebo centralizovaných výkupních míst.

Přestože poptávka po lokálních produktech v~dnešní době stále více roste\footfullcite{lokalni_potraviny_zajem, szif_2025}, způsoby jejich propagace zůstávají v~mnoha případech málo účinné.
Výsledkem je, že zákazníci často neví o~tom, jaké produkty jsou v~jejich blízkosti dostupné, a producenti nedokážou naplno využít potenciální zájem zákazníků.


\section{Motivace}
Typickým ilustračním příkladem je oblast včelařství a prodeje medu.
Producenti mají zpravidla dvě základní možnosti.
První možností je centrální výkup od včelařů, odkud se vykoupený med distribuuje do obchodů.
Výsledkem této možnosti je nižší podíl zisku pro samotného producenta.
Druhá možnost je lokální propagace, například pouze zobrazením nabídky cedulí u~domu, nebo osobním doporučením.
V tomto případě je však problémem nízké oslovení potenciálních zájemců, protože tyto metody mají spíše menší dosah.

Aplikace, která by producentům umožňovala prezentovat svoji lokální nabídku, by představovala vhodný nástroj jak pro problém s~propagací, tak pro problém s~nízkým ziskem pro tyto lokální výrobce.


\section{Existující aplikace a projekty}
V~rámci analýzy byly zkoumány existující aplikace a projekty, které se zaměřují na podporu lokálních farmářů a malých producentů.
Cílem bylo zjistit, zda již existuje nástroj zaměřený na malé nebo domácí producenty, který by splňoval požadavky na jednoduchou a přístupnou prezentaci nabídky jejich domácích produktů pro potenciální zájemce v~jejich okolí.

\subsection[Farmáři z~regionu]{Farmáři z~regionu}
Farmáři z~regionu je mapová aplikace, která sdružuje lokální farmáře a poskytuje informační katalog jejich produktů a provozoven.
Projekt umožňuje zobrazit jednotlivé farmy podle regionu a zprostředkovává základní údaje o~jejich nabídce.

Platforma je zaměřená především na oficiální farmáře a výrobce.
Obsahuje zemědělce, kteří o~zařazení projevili zájem v~rámci mapování produkce v~Jihomoravském kraji.
Další producenti se mohou přihlásit prostřednictvím formuláře a požádat o~přidání do aplikace, nejedná se ale o~otevřený marketplace, ve kterém by mohl kterýkoli uživatel okamžitě vytvářet vlastní obchody a spravovat svou nabídku.
Proces přidání do aplikace je řízený a je zaměřený spíše na oficiální farmáře a výrobce než na drobné domácí producenty.\footfullcite{metropolitni_brno_farmari_2025}

\subsection{Scuk.cz}
Scuk.cz je česká online platforma zaměřená na podporu lokálních farmářů a malých až středních producentů potravin.
Scuk klade důraz na kvalitu a původ produktů, a proto je vstup do platformy podmíněn schválením ze strany provozovatele.\footfullcite{scuk_napoveda_2025}

Platforma však není určena běžným domácím producentům, kteří chtějí jednoduše prezentovat svou nabídku.
Uživatelé si nemohou samostatně vytvářet obchody ani spravovat vlastní nabídku bez zapojení do složitého schvalovacího procesu.\footfullcite{scuk_kodex_2025}
Scuk navíc funguje jako nákupní platforma, obsahuje košík, objednávky a odběrná místa, takže jeho cílem je zprostředkování prodeje, nikoli pouze jednoduchá a přístupná prezentace nabídky.

Dalším rozdílem je způsob zobrazení prodejců.
Scuk využívá především katalog a filtrování produktů, zatímco mapové zobrazení není centrálním prvkem platformy.
To se liší od cíle této aplikace, která se soustředí právě na mapové zobrazení obchodů a snadnou prezentaci nabídky bez nutnosti formální registrace provozovny.

\subsection{Facebook skupiny a Facebook marketplace}

Facebook nabízí možnost vytváření skupin, které mohou být často zaměřeny kolem daných témat, o~které by mohl mít uživatel zájem.
Nabídky v~takových skupinách však postrádají možnost zobrazit si lokální nabídky a jejich přehledné zobrazení.
Příspěvky se navíc mohou ve skupině rychle ztratit nebo nemusí být aktuální.

Naopak Facebook Marketplace zase umožňuje dobré zobrazení produktu, ale je vhodný spíše k~jednotlivým inzerátům než k~dlouhodobé či měnící se nabídce.
Nenabízí však přehlednou nabídku producenta a ani není zaměřený na lokální prodej potravin a podobných produktů.

\subsection{Google Maps Places}
Google Maps sice umožňuje přehledné mapové zobrazení prostřednictvím služby Google Business Profile, ta je však primárně určena pro oficiálně registrované podniky a vyžaduje proces ověření.

Tento model může představovat bariéru pro drobné nebo příležitostné producenty, kteří nepůsobí jako formálně registrované provozovny.


\section{Výsledek analýzy}
Analýza dostupných nástrojů ukázala, že existující řešení jsou zpravidla zaměřena buď na oficiálně registrované producenty, nebo na zprostředkování samotného prodeje prostřednictvím centralizované platformy.
Jednoduchý nástroj umožňující drobným nebo domácím producentům samostatně vytvářet a spravovat prezentaci své nabídky s důrazem na mapové zobrazení a lokální dosah není v analyzovaných řešeních primárním cílem.

Tabulka~\ref{tab:platform-matrix} shrnuje srovnání analyzovaných platforem podle vybraných kritérií.
Hodnocení vyjadřuje přítomnost dané vlastnosti (\cmark), její částečné naplnění (\pmark) nebo její absenci (\xmark).

Tato skutečnost naznačuje prostor pro návrh aplikace zaměřené na snadnou prezentaci lokální nabídky bez nutnosti složitého schvalovacího procesu či zapojení do centralizovaného prodejního systému určené především pro domácí producenty.

\begin{table}[htbp]
    \centering
    \small
    \renewcommand{\arraystretch}{1.2}
    \caption{Hodnotící matice podpory drobných producentů}
    \label{tab:platform-matrix}

    \begin{threeparttable}

        \begin{tabularx}{\textwidth}{
                >{\raggedright\arraybackslash}p{4cm}
            *{5}{>{\centering\arraybackslash}X}
        }
    \toprule
    \textbf{Kritérium}
    & \textbf{Farmáři z regionu}
    & \textbf{Scuk.cz}
    & \textbf{Facebook}
    & \textbf{Google Maps}
    & \textbf{Navrhovaná aplikace} \\
    \midrule

    Samostatná registrace producenta
    & \pmark\tnote{a} & \pmark\tnote{b} & \cmark & \cmark & \cmark \\

    Zaměření na drobné producenty
    & \pmark & \pmark & \xmark & \xmark & \cmark \\

    Mapové zobrazení jako primární prvek
    & \cmark & \xmark & \xmark & \cmark & \cmark \\

    Nutnost schvalovacího procesu
    & \cmark & \cmark & \xmark & \pmark\tnote{c} & \xmark \\

    Zprostředkování přímého prodeje
    & \xmark & \cmark & \pmark\tnote{d} & \xmark & \xmark \\

    Jednoduchá prezentace lokální nabídky
    & \pmark & \xmark & \xmark & \pmark & \cmark \\

    \bottomrule
        \end{tabularx}

        \begin{tablenotes}
            \footnotesize
            \item[] \textbf{Legenda:}
            \cmark~Ano,
            \pmark~Částečně,
            \xmark~Ne.
            \item[a] Registrace probíhá formou žádosti o zařazení.
            \item[b] Vstup do platformy podléhá schvalovacímu procesu.
            \item[c] Založení profilu vyžaduje ověření podnikatelského subjektu.
            \item[d] Prodej probíhá formou jednotlivých inzerátů, nikoli dlouhodobé nabídky.
        \end{tablenotes}

    \end{threeparttable}

\end{table}