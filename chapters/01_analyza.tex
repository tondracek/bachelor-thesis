\chapter{Analýza}

V~oblasti lokálního prodeje potravin a dalších produktů se malé farmy, domácí pěstitelé a drobní výrobci často potýkají s~omezenými možnostmi, jak efektivně oslovit zákazníky ve svém okolí, bez nutnosti prodávat své produkty prostřednictvím velkých řetězců nebo centralizovaných výkupních míst.
Přestože poptávka po lokálních produktech v~dnešní době stále více roste, způsoby jejich propagace zůstávají v~mnoha případech málo účinné.
Výsledkem je, že zákazníci často neví o~tom, jaké produkty jsou v~jejich blízkosti dostupné, a producenti nedokážou naplno využít potenciální zájem zákazníků.


\section{Motivace}
Potřeba efektivnějšího digitálního nástroje vyplývá z~problémů, se kterými se drobní producenti se zájmem o~prodej běžně setkávají.
Možnosti propagace jsou často velmi omezené, například cedule u~silnice nebo osobní doporučení.
Tyto metody mají ale jen velmi malý dosah a neposkytují prostor pro dostatečnou prezentaci nabídky.

Typickým příkladem je oblast včelařství a prodeje medu.
Tady mají producenti většinou pouze dvě možnosti.
První možností je centrální výkup od včelařů, odkud se vykoupený med prodává do obchodů.
Výsledkem této možnosti je velmi nízká výkupní cena pro producenty a přirážka pro obchody.
Druhou možností je již zmíněné zobrazení nabídky, například cedulí u~domu.
Tady je opět problémem nízké oslovení potenciálních zájemců.

Aplikace, která by producentům umožňovala prezentovat svoji lokální nabídku, by představovala účinné řešení jak pro problém s~propagací, tak pro problém s~nízkým ziskem pro tyto lokální výrobce.


\section{Existující aplikace a projekty}
V~rámci analýzy byly zkoumány existující aplikace a projekty, které se zaměřují na podporu lokálních farmářů a malých producentů.
Cílem bylo zjistit, zda již existuje nástroj zaměřený na malé nebo domácí producenty, který by splňoval požadavky na jednoduchou a přístupnou prezentaci nabídky jejich domácích produktů pro potenciální zájemce v~jejich okolí.

\subsection[Farmáři z~regionu]{Farmáři z~regionu}
Farmáři z~regionu je mapová aplikace, která sdružuje lokální farmáře a poskytuje informační katalog jejich produktů a provozoven.
Projekt umožňuje zobrazit jednotlivé farmy podle regionu a zprostředkovává základní údaje o~jejich nabídce.

Platforma je zaměřená především na oficiální farmáře a výrobce.
Obsahuje zemědělce, kteří o~zařazení projevili zájem v~rámci mapování produkce v~Jihomoravském kraji.
Další producenti se mohou přihlásit prostřednictvím formuláře a požádat o~přidání do aplikace, nejedná se ale o~otevřený marketplace, ve kterém by mohl kterýkoli uživatel okamžitě vytvářet vlastní obchody a spravovat svou nabídku.
Proces přidání do aplikace je řízený a je zaměřený spíše na oficiální farmáře a výrobce než na drobné domácí producenty.\footfullcite{metropolitni_brno_farmari_2025}

\subsection{Scuk.cz}
Scuk.cz je česká online platforma zaměřená na podporu lokálních farmářů a malých až středních producentů potravin.
Scuk klade důraz na kvalitu a původ produktů, a proto je vstup do platformy podmíněn schválením ze strany provozovatele.\footfullcite{scuk_napoveda_2025}

Platforma však není určena běžným domácím producentům, kteří chtějí jednoduše prezentovat svou nabídku.
Uživatelé si nemohou samostatně vytvářet obchody ani spravovat vlastní nabídku bez zapojení do složitého schvalovacího procesu.\footfullcite{scuk_kodex_2025}
Scuk navíc funguje jako nákupní platforma, obsahuje košík, objednávky a odběrná místa, takže jeho cílem je zprostředkování prodeje, nikoli pouze jednoduchá a přístupná prezentace nabídky.

Dalším rozdílem je způsob zobrazení prodejců.
Scuk využívá především katalog a filtrování produktů, zatímco mapové zobrazení není centrálním prvkem platformy.
To se liší od cíle této aplikace, která se soustředí právě na mapové zobrazení obchodů a snadnou prezentaci nabídky bez nutnosti formální registrace provozovny.

\subsection{Facebook skupiny a Facebook marketplace}

Facebook nabízí možnost vytváření skupin, které mohou být často zaměřeny kolem daných témat, o~které by mohl mít uživatel zájem.
Nabídky v~takových skupinách však postrádají možnost zobrazit si lokální nabídky a zobrazení přehledné nabídky.
Příspěvky se navíc mohou ve skupině rychle ztratit nebo nemusí být aktuální.

Naopak Facebook Marketplace zase umožní dobré zobrazení produktu, ale je vhodný spíše k~jednotlivým inzerátům než k~dlouhodobé nabídce či měnící se nabídce.
Nenabízí však přehlednou nabídku producenta a ani není zaměřený na prodej potravin a podobných produktů.

\subsection{Google Maps Places}
Google Maps sice umožňuje přehledné mapové zobrazení, ale platforma je zaměřená na oficiální podniky, které si vytvoří firemní profil prostřednictvím služby Google Business Profile.
Provozovny musí mít ale splněny podmínky pro ověření firmy, které drobní domácí producenti nebo příležitostní pěstitelé zpravidla nesplňují a nemohou se tímto způsobem jednoduše prezentovat.


\section{Výsledek analýzy}
Analýza dostupných nástrojů ukázala, že ačkoliv existuje spousta projektů, které se snaží podpořit soukromé farmáře a malé až střední producenty, nástroj, který by skutečně usnadňoval menším, domácím producentům propagaci jejich nabídky produktů, na trhu chybí.

Toto vytváří prostor pro aplikaci, která bude umožňovat vytvářet nabídku domácích produktů a propagovat a zobrazovat ji pro lokální zájemce.
Producenty bude zobrazovat v~přehledném mapovém zobrazení, bez nutnosti komplikovaného zapojení se na platformu.