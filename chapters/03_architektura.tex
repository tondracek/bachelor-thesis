\chapter{Architektura kódu}


\section{Vertikální dělení vs horizontální dělení}
% features + ui (jako samostatná feature), moduly, packages, common, core


\section{Generický repozitář}
% firebasecollection anotace, generika, mappery

%Pro přístup k datům je použit generický repozitář, který odděluje doménovou logiku od konkrétní implementace datového zdroje.
% Zároveň tím předchází duplicitě kódu při práci s různými typy entit.
% Dovoluje mi tak pro každou novou funkcionalitu využít již existující základní implementaci a pro specifické potřeby jednotlivých entit pouze rozšířit o potřebné funkce.

% Zároveň umožňuje snadnou výměnu datového zdroje (svého jádra) bez nutnosti zásahu do zbytku aplikace.
% V repozitáři jsou využity generika pro práci s různými typy entit a mappery pro převod mezi datovými modely a doménovými modely.


\section{UC result wrapper}
% výhody/nevýhody, důvod, záměr, Success, Failure, funkcionální interface. Nadpis predelat na 


\section{UC třídy vs Jedna velká service třída}
% TODO: Nadpis predelat do cestiny (zejmena souslovi "service trida".
% výhody/nevýhody


\section{MVVM}
% TODO: Proc prave navrhovy vzor MVVM?