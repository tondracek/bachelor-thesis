\chapter{Architektura kódu}
Před začátkem implementace je důležité si rozmyslet a naplánovat architekturu kódu.
Dobře navržená architektura nám může ušetřit spoustu času a námahy v budoucnu, například při přidávání nových funkcí nebo opravování a předělávání stávajícího kódu.
Naopak špatně navržená architektura může vést k problémům, jako je těžko udržovatelný kód, problémy s testovatelností a obtížná spolupráce mezi vývojáři.

% Nativní android aplikace
% Využívá doporučený architektonický vzor MVVM (zdroj)
% Inspirace principy Clean Architecture (zdroj) a Hexagonal Architecture (zdroj). - TODO: dle chatgpt prý nemám zmiňovat...
% cíle architektury: čitelnost, oddělení odpovědností, rozšiřitelnost, opětovná použitelnost prvků


\section{Architektonický vzor MVVM}
% model-view-viewmodel
% oddělení odpovědností
% doporučený vzor pro nativní android aplikace
%
% podporované Android Frameworkem
%  - lifecycle awareness
%    - změny konfigurace (aplikace na pozadí, otočení obrazovky)
%    - životní cyklus aktivit (onCreate, onResume, onPause)
%  - hilt integrace (injekce závislostí, scoping, viewmodel store owner)
%
% reaktivní programování - Flow
%
% Ve zkratce:
%  - role ViewModelu:
%    - správa stavu obrazovky
%    - reakce na uživatelské události
%  - role View:
%    - zobrazování dat
%    - předávání uživatelských událostí ViewModelu
%  - role Model:
%    - reprezentace dat a doménové logiky
%    - komunikace s repozitáři a datovými zdroji

% TODO: zmínit immutabilty princip?


\section{Struktura projektu podle funkčních celků}
% *Feature-based struktura projektu*

% každá feature obsahuje:
% - domain
% - data
% - případně di

% konkrétní features (+ vysvětlení):
% - auth
% - core - DomainResult (viz. \ref{sec:domain_result}), ...
% - common
% - shop, review, user, ...
% - ui - dle obrazovek/samostatných celků

% Zdůraznit přínos:
% - lepší orientace v kódu
% - omezení provázanosti
% - možnost paralelního vývoje (možnost kompletně přepsat jednu feature bez vlivu na ostatní)


\section{Oddělení doménové a datové vrstvy}

% Domain
%  - doménové modely
%  - use case třídy
%  - rozhraní repozitářů
% "Komunikace s backendem probíhá přes rozhraní definovaná v doménové vrstvě."

% Data
%  - implementace repozitářů
%  - datové modely
%  - práce s Firestore, Firebase Auth, Firebase Storage, ...


\section{Zpracování výsledků a chyb v doménové vrstvě}
% DomainResult a DomainError
% princip...
% výhody...

% Zdůrazni:
%  - návratová hodnota use-casů a repozitářů
%  - nutí volajícího explicitně řešit chyby


\section{Tok dat v aplikaci}

% příklad toku dat:
% 1. uživatelská akce v UI
% 2. ViewModel zpracuje událost
% 3. volání use-casu, doménová logika
% 4. volání repozitáře
% 5. výsledek jako DomainResult z repozitáře (respektive DomainError v případě chyby)
% 6. aktualizace StateFlow (respektive zpracování chyby)
% 7. UI se automaticky přerenderuje

% TODO: zmínit immutabilty princip?


\section{Vkládání závislostí (Dependency Injection)}
% Hilt
% výhody DI
%  - lepší testovatelnost
%  - méně kódu pro vytváření závislostí
%  - pro viewmodely, use-casy, repozitáře


\section{Shrnutí architektury}
% architektura odpovídá rozsahu bakalářské práce
% ověřena reálným provozem aplikace
% umožnila úspěšnou implementaci a publikaci aplikace
% zjednodušila vývoj, údržbu kódu, přidávání nových funkcí a opravy chyb
%
% "Zvolená architektura se v praxi osvědčila jako přehledná, rozšiřitelná a vhodná pro další rozvoj aplikace."
